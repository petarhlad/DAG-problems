%%%%%%%%%%%%%%%%%%%%%%%%%%%%%%%%%%%%%%%%%
% Short Sectioned Assignment
% LaTeX Template
% Version 1.0 (5/5/12)
%
% This template has been downloaded from:
% http://www.LaTeXTemplates.com
%
% Original author:
% Frits Wenneker (http://www.howtotex.com)
%
% License:
% CC BY-NC-SA 3.0 (http://creativecommons.org/licenses/by-nc-sa/3.0/)
%
%%%%%%%%%%%%%%%%%%%%%%%%%%%%%%%%%%%%%%%%%

%----------------------------------------------------------------------------------------
%	PACKAGES AND OTHER DOCUMENT CONFIGURATIONS
%----------------------------------------------------------------------------------------

\documentclass[paper=a4, fontsize=11pt]{scrartcl} % A4 paper and 11pt font size

\usepackage[T1]{fontenc} % Use 8-bit encoding that has 256 glyphs
%\usepackage{fourier} % Use the Adobe Utopia font for the document - comment this line to return to the LaTeX default
\usepackage[english]{babel} % English language/hyphenation
\usepackage{amsmath,amsfonts,amsthm,amssymb} % Math packages

\usepackage{sectsty} % Allows customizing section commands
%\allsectionsfont{\centering \normalfont\scshape} % Make all sections centered, the default font and small caps
\allsectionsfont{\centering}

\usepackage{fancyhdr} % Custom headers and footers
\pagestyle{fancyplain} % Makes all pages in the document conform to the custom headers and footers
\fancyhead{} % No page header - if you want one, create it in the same way as the footers below
\fancyfoot[L]{} % Empty left footer
\fancyfoot[C]{} % Empty center footer
\fancyfoot[R]{\thepage} % Page numbering for right footer
\renewcommand{\headrulewidth}{0pt} % Remove header underlines
\renewcommand{\footrulewidth}{0pt} % Remove footer underlines
\setlength{\headheight}{13.6pt} % Customize the height of the header

%\numberwithin{equation}{section} % Number equations within sections (i.e. 1.1, 1.2, 2.1, 2.2 instead of 1, 2, 3, 4)
%\numberwithin{figure}{section} % Number figures within sections (i.e. 1.1, 1.2, 2.1, 2.2 instead of 1, 2, 3, 4)
%\numberwithin{table}{section} % Number tables within sections (i.e. 1.1, 1.2, 2.1, 2.2 instead of 1, 2, 3, 4)

\setlength\parindent{0pt} % Removes all indentation from paragraphs - comment this line for an assignment with lots of text

\usepackage{caption}

\usepackage{algorithm}
\usepackage[noend]{algorithmic}

%\floatname{algorithm}{Procedure}
\renewcommand{\algorithmicrequire}{\textbf{Input:}}
\renewcommand{\algorithmicensure}{\textbf{Output:}}

\newtheorem{mydef}{Definition}
\theoremstyle{plain}
\newtheorem{lemma}{Lemma}


\usepackage{graphicx}
\graphicspath{{../images/}}



%----------------------------------------------------------------------------------------
%	TITLE SECTION
%----------------------------------------------------------------------------------------

\newcommand{\horrule}[1]{\rule{\linewidth}{#1}} % Create horizontal rule command with 1 argument of height

\title{	
\normalfont \normalsize 
\textsc{UPC - Discrete and algorithmic geometry} \\ [25pt] % Your university, school and/or department name(s)
\horrule{0.5pt} \\[0.4cm] % Thin top horizontal rule
\huge Problem sheet 2  \\ % The assignment title
\horrule{2pt} \\[0.5cm] % Thick bottom horizontal rule
}

\author{Simon Van den Eynde \\ Petar Hlad Colic} % Your name

\date{\normalsize\today} % Today's date or a custom date

\begin{document}

\maketitle % Print the title

\section{Nonrational Pentagon}
Given the incidences we have given, we should show that the inner pentagon is regular, then we can use http://mathworld.wolfram.com/Pentagon.html to show the construction cannot be realised with rational coordinates.

\section{6.9: enumerate 4-polytopes with 7 vertices}
Things we can use: \\
a) Theorem 6.19 in Ziegler says that a Gale diagram represents a polytope iff every cocircuit has at least 2 positive elements.\\

b) Their is a cyclic symmetry (+,+,-,-,-,-,+)= (+,-,-,-,-,+,-)\\

c) We should also consider coinciding points (I don't know if we should think about special points as well (pyramids)).

\section{6.15: 2 different 2-neighborly 4-polytopes}
In $C_{4}(7)$ there are no tetrahedra, in the second Gale diagram we can find some, for example since CH(137) and CH(2) intersect, 4568 is a facet, since it's 3-dimensional it's a simplex.
 
\section{6.17: analyse gale diagram}
Faces are:
235678

1268
1267
3457
3458

14568
14567
12347
12348

If we have the same circuits, we have the same intersection so both 7 and 8 should lie on the line 23 and on the line 56, since these lines are different and the intersection is 1 point, they lie on the same point.

Then 2356 lie in the same plane (I'm going to think a bit about this).

Octahedron cannot be prescribed and the oriented matroid is not rigid, I don't know yet.

\end{document}











