%%%%%%%%%%%%%%%%%%%%%%%%%%%%%%%%%%%%%%%%%
% Short Sectioned Assignment
% LaTeX Template
% Version 1.0 (5/5/12)
%
% This template has been downloaded from:
% http://www.LaTeXTemplates.com
%
% Original author:
% Frits Wenneker (http://www.howtotex.com)
%
% License:
% CC BY-NC-SA 3.0 (http://creativecommons.org/licenses/by-nc-sa/3.0/)
%
%%%%%%%%%%%%%%%%%%%%%%%%%%%%%%%%%%%%%%%%%

%----------------------------------------------------------------------------------------
%	PACKAGES AND OTHER DOCUMENT CONFIGURATIONS
%----------------------------------------------------------------------------------------

\documentclass[paper=a4, fontsize=11pt]{scrartcl} % A4 paper and 11pt font size

\usepackage[T1]{fontenc} % Use 8-bit encoding that has 256 glyphs
%\usepackage{fourier} % Use the Adobe Utopia font for the document - comment this line to return to the LaTeX default
\usepackage[english]{babel} % English language/hyphenation
\usepackage{amsmath,amsfonts,amsthm,amssymb} % Math packages

\usepackage{sectsty} % Allows customizing section commands
%\allsectionsfont{\centering \normalfont\scshape} % Make all sections centered, the default font and small caps
\allsectionsfont{\centering}

\usepackage{fancyhdr} % Custom headers and footers
\pagestyle{fancyplain} % Makes all pages in the document conform to the custom headers and footers
\fancyhead{} % No page header - if you want one, create it in the same way as the footers below
\fancyfoot[L]{} % Empty left footer
\fancyfoot[C]{} % Empty center footer
\fancyfoot[R]{\thepage} % Page numbering for right footer
\renewcommand{\headrulewidth}{0pt} % Remove header underlines
\renewcommand{\footrulewidth}{0pt} % Remove footer underlines
\setlength{\headheight}{13.6pt} % Customize the height of the header

%\numberwithin{equation}{section} % Number equations within sections (i.e. 1.1, 1.2, 2.1, 2.2 instead of 1, 2, 3, 4)
%\numberwithin{figure}{section} % Number figures within sections (i.e. 1.1, 1.2, 2.1, 2.2 instead of 1, 2, 3, 4)
%\numberwithin{table}{section} % Number tables within sections (i.e. 1.1, 1.2, 2.1, 2.2 instead of 1, 2, 3, 4)

\setlength\parindent{0pt} % Removes all indentation from paragraphs - comment this line for an assignment with lots of text

\usepackage{caption}

\usepackage{algorithm}
\usepackage[noend]{algorithmic}

%\floatname{algorithm}{Procedure}
\renewcommand{\algorithmicrequire}{\textbf{Input:}}
\renewcommand{\algorithmicensure}{\textbf{Output:}}

\newtheorem{mydef}{Definition}
\theoremstyle{plain}
\newtheorem{lemma}{Lemma}


\usepackage{graphicx}
\graphicspath{{../images/}}



%----------------------------------------------------------------------------------------
%	TITLE SECTION
%----------------------------------------------------------------------------------------

\newcommand{\horrule}[1]{\rule{\linewidth}{#1}} % Create horizontal rule command with 1 argument of height

\title{	
\normalfont \normalsize 
\textsc{UPC - Discrete and algorithmic geometry} \\ [25pt] % Your university, school and/or department name(s)
\horrule{0.5pt} \\[0.4cm] % Thin top horizontal rule
\huge Problem sheet 2  \\ % The assignment title
\horrule{2pt} \\[0.5cm] % Thick bottom horizontal rule
}

\author{Simon Van den Eynde \\ Petar Hlad Colic} % Your name

\date{\normalsize\today} % Today's date or a custom date

\begin{document}

\maketitle % Print the title

\section{Matousek}
\subsection{2}
\subsubsection{Problem description}
We will use Lemma 5.1.2 (Duality preserves incidences) ii) in Matousek: 
Let $p$ be a point of $\mathbb{R}^{d}$ distinct from the origin and let $h$ be a hyperplane in $\mathbb{R}^d$ not containing the origin. Let $h^-$ stand for the closed half-space bounded by $h$ and containing the origin, while $h^+$ denotes the other closed half-space bounded by $h$. That is, if $h=\{x\in\mathbb{R}^d: \langle a,x\rangle=1\}$, then $h^-=\{x\in\mathbb{R}^d: \langle a,x\rangle\leq1\}$.
Then $p\in h^- \iff D_0(h)\in D_0(p)^-$.\\

Let us consider a pentagon which contains the origin. Let $v_i = D_0(l_i)$, where $l_i$ is the line containing the side $a_ia_{i+1}$. Then the points dual to the lines intersecting the pentagon $a_1a_2\ldots a_5$ fill exactly the exterior of the convex pentagon $P_{ex} = v_1v_2\ldots v_5$.

\subsection{solution}
So take a point $p$ outside $P_{ex}$. Because $P_{ex}$ is convex there exists an edge $v_iv_{i+1}$ (we assume here that $v_{5+1}=v_1$) with supporting line $h$, such that $p$ lies in the halfplane $h^+$ (not containing the origin). Because duality preserves incidences we find $D_0(h)\in D_0(p)^+$. Now since $D_0(h)=D_0(D_0(a_{i+1}))=a_{i+1}$, we find that the line segment $[0,a_{i+1}]\cap D_0(p)\neq \varnothing$. So $D_0(p)$ intersects $P_ex$.

Why is $v_iv_{i+1}=D_0(a_{i+1})$?
\begin{align*}
v_i&=D_0(a_{i}a_{i+1}),& v_j=D_0(a_{i+1}a_{i+2})\\
\implies &D_0(v_i)=a_ia_{i+1},& D_0(v_j)=a_{i+1}a_{i+2}\\
\implies &D_0(v_i)D_0(v_j)=a_{i+1}\\
\implies &D_0(a_{i+1})=v_iv_j
\end{align*}

Now analogous to the first part we can take a point $p$ inside $P_ex$ then we find $v_{i}v_{i+1}$ with a supporting line $h$ such that $p\in h^-$. We find $D_0(h)=a_{i+1}\in D_0(p)^-$ So that $D_0(p)$ intersects the line $a_{i+1}$ outside the pentagon. Because $D_0(p)$ is perpendicular to $a_{i+1}$ it will never intersect the convex polygon

\subsection{3}
\begin{align*}
X^{*} &= \{y\in \mathbb{R}^{d} : \langle x,y\rangle \leq 1, \forall x \in X\}\\
X^{**} &= \{y\in \mathbb{R}^{d} : \langle x,y\rangle \leq 1, \forall x \in X^{*}\}
\end{align*}
 
Now because $\forall x\in X,\forall y \in X^{*} : \langle x,y\rangle \leq 1 \implies x\in X^{**}$. Also clearly $0\in X^{**}$. Since $X^{**}$ closed and convex we find $conv(X\cup 0)\subset X^{**}$.

The separation theorem says that for a closed set $Z$: $conv(Z)=\bigcap$(all closed halfspaces that contain Z).
So $conv(X\cup 0)$ is the intersection of all closed halfspace that contain $0$ and $X$.

\section{$C_4(7)$}
\subsection{(a)}
We will first calculate the $f$-vector of $C_4(7)$.
\begin{itemize}
\item $f_0=7$
\item $f_1= \binom{7}{2} = 21$, because $C_4(7)$ is neighborly
\item $f_3 = 14$, we counted this in class, using Gale's evenness criterium.
\end{itemize}

Define $h_k = \sum_{i\geq k}^{d} f_{i}(-1)^{(i-k)}\binom{k}{i} $, with $f_{-1}=f_{d}=1$. Then the Dehn-Sommerville equations learn us that $h_i=h_{d-i}$.\\
We find $h_0=f_0-f_1+f_2-f_3+f_4=7-21+f_2-14+1=f_2-27$ and $h_4=f_4=1$. Now $h_0=h_4\implies f_3=28$.

So $f(C_4(7))=(7,21,28,14)$. And we find $f(C_4(7)^\Delta)=(14,28,21,7)$
\section{$24$-cell}


\end{document}
